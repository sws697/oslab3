\documentclass[UTF8]{ctexart}
\usepackage{amsmath}
\usepackage{amssymb}
\usepackage{extarrows}
\usepackage{bm}
\usepackage{listings}
\usepackage{xcolor}
\usepackage{graphicx}
\definecolor{codegreen}{rgb}{0,0.6,0}
\definecolor{codegray}{rgb}{0.5,0.5,0.5}
\definecolor{codepurple}{rgb}{0.58,0,0.82}
\definecolor{backcolour}{rgb}{0.95,0.95,0.92}

\lstdefinestyle{mystyle}{
    backgroundcolor=\color{backcolour},   
    commentstyle=\color{codegreen},
    keywordstyle=\color{magenta},
    numberstyle=\tiny\color{codegray},
    stringstyle=\color{codepurple},
    basicstyle=\ttfamily\footnotesize,
    breakatwhitespace=false,         
    breaklines=true,                 
    captionpos=b,                    
    keepspaces=true,                 
    numbers=left,                    
    numbersep=5pt,                  
    showspaces=false,                
    showstringspaces=false,
    showtabs=false,                  
    tabsize=2
}

\lstset{style=mystyle}
\title{操作系统实验三}
\author{汪航 2021211114}
\date{\today}
\begin{document}
\maketitle
\section{实验内容}
本次实验共编写两道程序,第一个程序使用POSIX标准的pthread库编写了多线程执行求平均数、求最大值、求最小值的程序;第二个程序使用了pthread库编写了矩阵乘程序。
\section{实验环境}
Linux操作系统,gcc编译
\section{程序一}
\subsection{主要内容}
以多个数字为输入,使用pthread库编写求这多个数字的平均数、最大值、最小值的多线程程序。

\paragraph{使用API:}pthread\_attr\_init, pthread\_create, pthread\_join
\subsection{程序设计思路}
开辟三个线程, 三个线程分别用于计算平均数、最大值、最小值,平均数、最大值、最小值设置为全局变量,父子线程之间通过全局变量传递信息,最终结果由父线程打印。\\
程序实现:
\begin{lstlisting}[language=C]
#include <pthread.h>
#include <stdio.h>
#include <stdlib.h>
int minimum = __INT32_MAX__;
int maximum = -__INT32_MAX__ - 1;
int average = 0;
int numbercount = 0;
void *claverage(char *param[]);
void *clminimum(char *param[]);
void *clmaximum(char *param[]);
int main(int argc, char *argv[])
{
    pthread_t workers[3];
    pthread_attr_t attr;
    numbercount = argc - 1;

    pthread_attr_init(&attr);

    pthread_create(&workers[0], &attr, claverage, argv);
    pthread_create(&workers[1], &attr, clminimum, argv);
    pthread_create(&workers[2], &attr, clmaximum, argv);
    for (int i = 0; i < 3; i++)
    {
        pthread_join(workers[i], NULL);
    }
    printf("The average value is %d\n", average);
    printf("The minumum value is %d\n", minimum);
    printf("The maximum value is %d\n", maximum);
    return 0;
}
void *claverage(char *param[])
{
    for (int i = 1; i <= numbercount; i++)
    {
        average += atoi(param[i]);
    }
    average = average / numbercount;
}
void *clminimum(char *param[])
{
    for (int i = 1; i <= numbercount; i++)
    {
        int k = atoi(param[i]);
        if (k < minimum)
        {
            minimum = k;
        }
    }
}
void *clmaximum(char *param[])
{
    for (int i = 1; i <= numbercount; i++)
    {
        int k = atoi(param[i]);
        if (k > maximum)
        {
            maximum = k;
        }
    }
}
\end{lstlisting}
\subsection{程序测试结果及测试结果分析}
\paragraph{输入:}./lab3 90 81 78 95 79 72 85
\paragraph{输出:}The average value is 82\\
The minumum value is 72\\
The maximum value is 95\\
\paragraph{输入:}./lab3 29 281 4 2 44 2 45 31 563
\paragraph{输出:}The average value is 111\\
The minumum value is 2\\
The maximum value is 563\\
\paragraph{输入:}./lab3 -2 3 -212 213 -2 -6
\paragraph{输出:}The average value is -1\\
The minumum value is -212\\
The maximum value is 213\\
\begin{figure}[htbp]
    \centering
    \includegraphics{1.png}
    \caption{运行结果}
\end{figure} 
\begin{figure}[htbp]
    \centering
    \includegraphics{2.png}
    \caption{运行结果}
\end{figure} 
经分析,该程序能够正常执行期望的求平均数、最大值、最小值功能。
\section{程序二}
\subsection{主要内容}
给定矩阵A,B,求A*B的结果C.

\paragraph{使用API:}pthread\_attr\_init, pthread\_create, pthread\_join
\subsection{程序设计思路}
A为M*K矩阵, B为K*N矩阵,则对每一个\(C_{i,j}\)都可以开辟一个线程进行\(C_{i,j}\)的计算,计算\(C_{i,j}\)的线程执行了\(\sum_{n=1}^{k}A_{i,n}B_{n,j}\)的计算。\\
程序实现:
\begin{lstlisting}[language=C]
#include<pthread.h>
#include<stdio.h>
#include<stdlib.h>
#define M 3
#define K 2
#define N 3
struct v
{
    int row;
    int column;
};

int A[M][K] = {{1, 4}, {2, 5}, {3, 6}};
int B[K][N] = {{8,7,6},{5,4,3}};
int C[M][N]={0};
void *runner(void * param)
{

    int row = ((struct v *)param)->row;
    int column = ((struct v *)param)->column;
    for (int i = 0; i < K;i++)
    {
        C[row][column] += A[row][i] * B[i][column];
    }
}
int main()
{
    pthread_attr_t attr;
    pthread_attr_init(&attr);
    pthread_t workers[M * N];
    for (int i = 0; i < M; i++)
    {
        for (int j = 0; j < N;j++)
        {
            struct v *data = (struct v *)malloc(sizeof (struct v));
            data->row = i;
            data->column = j;
            pthread_create(&workers[i * N + j], &attr, runner, data);
        }
    }
    for (int i = 0; i < M;i++)
    {
        for (int j = 0; j < N;j++)
        {
            pthread_join(workers[i * N + j], NULL);
        }
    }
    for (int i = 0; i < M;i++)
    {
        for (int j = 0; j < N;j++)
        {
            printf("%d ", C[i][j]);
        }
        printf("\n");
    }
}
\end{lstlisting}
\subsection{程序测试结果及测试结果分析}
\paragraph{输入:}A:\{\{1, 4\}, \{2, 5\}, \{3, 6\}\}, B:\{\{8,7,6\},\{5,4,3\}\};
\paragraph{输出:} 
 
28 23 18 \\
41 34 27 \\
54 45 36\\
\begin{figure}[htbp]
    \centering
    \includegraphics[width=9.5cm,height=10cm]{3.png}
    \caption{运行结果}
\end{figure} 
经分析,该程序能够正常执行期望的矩阵相乘功能。
\end{document}